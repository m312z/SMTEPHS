\section{Related Work}\label{sec:related_work}

In this section we discuss the related work. First in section~\ref{sec:rel_sat} we describe the preceding work in planning as satisfiability, and then in section~\ref{sec:rel_hyb} we place our work with resepect to other approaches to planning in hybrid domains.

\subsection{Planning as Satisfiability}\label{sec:rel_sat}

Planning as Satisfiability was pioneered by Kautz and Selman, beginning with a translation from Planning into propositional satisfiablity (SAT)~\cite{kau92}, and the planner SATPlan~\cite{kau06}. Since then there have been many contributions improving the effectiveness of planning as SAT, including alternate encodings of the state or transition relation; the embedding of additional planning-specific knowledge such as heuristic evaluation; or planning-specific improvements to the SAT solver. Many of these ideas are orthogonal to the choice of representation between SAT and SMT.
%
There are relatively few approaches to planning as SMT. The dReach planner plans a subset of PDDL+ using the dReal SMT solver for ODEs, and {\sc TM-LPSAT} uses an encoding of propositional and numeric variables with linear constraints solved by the LPSAT engine. However, neither of these approaches produce SMT problems that can be solved by general SMT solvers.

Lifted causal encodings, first introduced by Kautz et al.~\cite{kau96a}, inspired by the lifted version of the SNLP causal link planner of McAllester and Rosenblitt~\cite{mca91} differ from the state-based encodings in that there is no proper notion of a state. The lifted encoding encodes a propositional planning problem as lifted SAT, which is then reduced to SAT. The encoding includes an assignment of action to plan steps, and the a valid causal ordering between plan steps. While it proved less effective in propositional planning with SAT, this work is applicable using the first-order expressions of SMT in a partially-ordered happening-based encoding.

Rintanen et al. have greatly advanced the state-in-the-art for planning as satisfiability, including new semantics for plan steps~\cite{rin06}; embedding planning heuristics in the SAT solver~\cite{rin10a}, such as the "helpful actions" heuristic in the planner Madagascar~\cite{rin10}, and other top-level search strategies that improve over the iterative deepening used by SATPlan and SMTPlan~\cite{rin04}. A shared purpose of these advancements is to help the SAT solver scale to the large discrete search space present in most planning problems. These approaches are orthogonal to the choice of encoding formalism (SAT or SMT), and can be applied directly in SMTPlan.

The planner ITSAT~\cite{ran15} is a SAT-based planner for non-numeric temporal planning problems. The planner extends the step semantics introduced by Rintanen et al.~\cite{rin06} to temporally abstract the problem without losing the ability to express concurrent activities.

\subsection{Planning in Hybrid Domains}\label{sec:rel_hyb}

Various techniques and tools have been proposed to deal with hybrid domains. ZENO~\cite{zeno} is a planner which can handle actions occurring over extended intervals of time. ZENO is able to reason about goals with deadlines, piece-wise linear continuous change, external events and to a limited extent, simultaneous actions.

More recent approaches in this direction have been proposed by Bologomov et al.~\cite{bogomolov14}, where the close relationship between hybrid planning domains and hybrid automata is explored, and \cite{bryce} where hybrid domains are handled using SMT.
%
dReach~\cite{bryce}, a planner for hybrid systems uses the dReal solver~\cite{gao12}, a non-linear SMT solver that uses its own theory of ODEs. Input is provided in the language of dReach as opposed to PDDL+, and hybrid problems have to be manually encoded. This language can only handle a restricted subset of the language features contained in PDDL+. In particular, it cannot handle exogenous events. 

More similar to the approach of SMTPlan is the planner TM-LPSAT~\cite{TM-LPSAT}. The planner is able to solve problems with atomic and durative actions, processes, events, and linear change. TM-LPSAT uses a happening-based encoding, containing propositional and numeric variables and linear constraints. This is solved by the LPSAT constraint engine. While SMTPlan is able to handle polynomial non-linear change, TM-LPSAT is restricted to continuous linear change; the LPSAT solver requires only linear constraints, and the encoding does not account for the zero-crossing problem introduced by non-linear change.

% this is the abstract of the TM-LPSAT paper
%This planner is formed of three separate stages . Initially a representation of the domain and problem file are compiled into a propositional system. This propositional system is formed of the set of propositional variables and linear constraints over numerical variables. In order to find the solution for this system as the next stage, the LPSAT engine is used. In the final stage, a corresponding plan to the solution resulted by the LPSAT is produced.

Many works have been proposed in the model checking and control communities to handle hybrid systems~\cite{hycomp,nuxmv,smthybrid,pappas,maly}, including sampling-based planners \cite{rrt,sampl}. Another related direction is  \textit{falsification} of hybrid systems~\cite{falsif} (i.e., guiding the search towards the error states, that can be easily cast as a planning problem). However, while all these works aim to address a similar problem, they cannot be used to handle PDDL+ models. Bogomolov et al.~\cite{bogomolov14,bogomolov15} are working towards a formal translation between PDDL+ and standard hybrid automata, but so far only an over-approximation has been defined, which allows the use of those tools only for proving plan non-existence.

To date, the only viable approach to general PDDL+ planning is via discretisation. UPMurphi~\cite{upmurphi}, which implements the discretise-and-validate approach, is able to deal with the full range of PDDL+ features. Discretise and validate works by first discretising time into small steps, solving the problem, and validating the result against the original continuous domain. If the plan is not valid with respect to the continuous semantics, then a finer discretisation is generated and the process iterates. However, UPMurphi performs blind search, which limits its scalability.

More recent work in this direction is the planner, DiNo. Similar to UPMurphi, DiNo is also based on the discretise-and-validate approach and uses the novel Staged Relaxed Planning Graph+ (SRPG+) heuristic~\cite{piotrowski2016heuristic} to help cope with the scalability issues faced by UPMurphi.

Considering the related works mentioned above, the SMT encoding is able to capture all features of PDDL+ and works by directly translating standard PDDL+ domain and problem files. Furthermore, it correctly captures the \textit{must} semantics of PDDL+ (which constrains how processes and events interact with each other and with actions). Also, SMTPlan models the precise semantics of $\epsilon$-separation of effects and action preconditions~\cite{pddl+}.

%~\cite{optop,kongming,coles12}