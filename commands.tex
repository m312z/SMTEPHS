\def\memout{\hspace*{\fill}--\hspace*{\fill}}
\newcommand{\eg}{e.\,g.}
\newcommand{\ie}{i.\,e.}
\newtheorem{proposition}{Proposition}
\newtheorem{theorem}{Theorem}
\newtheorem{lemma}{Lemma}
\newtheorem{example}{Example}

\newtheoremstyle{my_def}% name
  {5pt}%      Space above
  {5pt}%      Space below
  {}%         Body font
  {}%         Indent amount (empty = no indent, \parindent = para indent)
  {\bfseries}% Thm head font
  {.}%        Punctuation after thm head
  {.5em}%     Space after thm head: " " = normal interword space;
        %       \newline = linebreak
  {{#1} {#2} (\thmnote{#3})}%         Thm head spec (can be left empty, meaning `normal')
\theoremstyle{my_def}
\newtheorem{definition}{Definition}

\newcommand{\astar}{\ensuremath{\textup{A}^*}}
\newcommand{\idastar}{\ensuremath{\textup{IDA}^*}}
\newcommand{\ha}{\ensuremath{\mathcal{H}}}

\newcommand{\R}{\ensuremath{\mathcal{R}}}
\newcommand{\var}{\ensuremath{\mathit{Var}}}
\newcommand{\loc}{\ensuremath{\mathit{Loc}}}
\newcommand{\relc}{\ensuremath{\mathit{Flow}}}
\newcommand{\reld}{\ensuremath{\mathit{Trans}}}
\newcommand{\inv}{\ensuremath{\mathit{I}}}
\newcommand{\Input}{\ensuremath{\mathcal{U}}}
\newcommand{\RealUniverse}{\ensuremath{\mathbb{R}^n}}
\newcommand{\update}{\ensuremath{\xi}}
\newcommand{\guard}{\ensuremath{g}}
\newcommand{\HyperBox}{\ensuremath{\mathcal{B}}}
\newcommand{\Vector}[1]{\ensuremath{\mathbf{#1}}}
\newcommand{\PostC}[2]{post_c(#1,#2)}
\newcommand{\PostD}[2]{post_d(#1,#2)}
\newcommand{\sync}{\ensuremath{\mathit{Sync}}}
\newcommand{\lab}{\ensuremath{\mathit{label}}}
\newcommand{\dur}{\ensuremath{\mathit{dur}}}
\newcommand{\lockstart}{\ensuremath{\overline{\mathit{lock}_a^\mathit{start}}}}
\newcommand{\lockend}{\ensuremath{\overline{\mathit{lock}_a^\mathit{end}}}}
\newcommand{\releasestart}{\ensuremath{\overline{\mathit{release}_a^\mathit{start}}}}
\newcommand{\releaseend}{\ensuremath{\overline{\mathit{release}_a^\mathit{end}}}}
\newcommand{\lock}{\ensuremath{\overline{\mathit{lock}_a}}}
\newcommand{\release}{\ensuremath{\overline{\mathit{release}_a}}}
\newcommand{\lockstartend}{\ensuremath{\overline{\mathit{lock}_a^\mathit{start/end}}}}
\newcommand{\releasestartend}{\ensuremath{\overline{\mathit{release}_a^\mathit{start/end}}}}

\newcommand{\ContState}{\mathbf{x}}
\newcommand{\ContStateFunc}{\mathbf{X}}
\newcommand{\DiscStateFunc}{L}
\newcommand{\ContStateStart}{\mathbf{X}_{start}}
\newcommand{\ContStateEnd}{\mathbf{X}_{end}}
\newcommand{\run}{\rho}
\newcommand{\traj}{\tau}

\newcommand{\network}{\ensuremath{\mathcal{N}}}
\renewcommand{\epsilon}{\ensuremath{\varepsilon}}

\newcommand{\dom}{\ensuremath{\mathit{Dom}}}
\newcommand{\prob}{\ensuremath{\mathit{Prob}}}
\newcommand{\fs}{\ensuremath{\mathit{Fs}}}
\newcommand{\rs}{\ensuremath{\mathit{Rs}}}
\newcommand{\as}{\ensuremath{\mathit{As}}}
\newcommand{\ps}{\ensuremath{\mathit{Ps}}}
\newcommand{\es}{\ensuremath{\mathit{Es}}}
\newcommand{\arity}{\ensuremath{\mathit{arity}}}
\newcommand{\os}{\ensuremath{\mathit{Os}}}
\newcommand{\init}{\ensuremath{\mathit{Init}}}
\newcommand{\Hinit}{\ensuremath{\mathit{Init}}}

\newcommand{\headingtimeaction}{{\bf Time} \qquad \= {\bf Action}\\[0.8ex]}
\newcommand{\headingtimehappening}{{\bf Time} \qquad \= {\bf Happening}\\[0.8ex]}
\newcommand{\headingtimedetails}{{\bf Time} \qquad \= {\bf Details}\\[0.8ex]}
\newcommand{\atime}[1]{{\bf #1:}}
\newcommand{\action}[1]{{\sf #1}}
\newcommand{\exprn}[1]{{\sf #1}}
\newcommand{\fexprn}[1]{{\small {\bf #1}}}
\newcommand{\condeffmon}[1]{#1 {\it - conditional effect monitor}}
\newcommand{\updatectspne}{{\it Update of changing Primitive Numerical Expressions}}
\newcommand{\actionstart}[1]{#1 {\it - start}}
\newcommand{\actionend}[1]{#1 {\it - end}}
\newcommand{\actioninv}[1]{{\it Invariant for } #1}
\newcommand{\checkhappening}{Checking Happening... }
\newcommand{\eventtriggered}{{\bf Event triggered!}}
\newcommand{\listrow}[1]{\begin{minipage}[t]{11.5cm} #1 \end{minipage}}
\newcommand{\listrowg}[1]{\begin{minipage}[t]{10cm} #1 \end{minipage}}
\newcommand{\happeningOK}{...OK!}
\newcommand{\notOK}{...NOT OK!}
\newcommand{\aprocessactivated}[1]{\listrow{{\it Activated process #1} }}
\newcommand{\aprocessunactivated}[1]{\listrow{{\it Unactivated process #1} }}
\newcommand{\aeventtriggered}[1]{\listrow{{\it Triggered event #1} }}
\newcommand{\assignment}[3]{\listrow{Updating \fexprn{#1} (#2) by #3 assignment.}}
\newcommand{\assignmentcts}[3]{\listrow{Updating \fexprn{#1} (#2) by #3 for continuous update.}}
\newcommand{\increase}[3]{\listrow{Increasing \fexprn{#1} (#2) by #3.}}
\newcommand{\decrease}[3]{\listrow{Decreasing \fexprn{#1} (#2) by #3.}}
\newcommand{\scaleup}[3]{\listrow{Scaling up \fexprn{#1} (#2) by a factor of #3.}}
\newcommand{\scaledown}[3]{\listrow{Scaling down \fexprn{#1} (#2) by a factor of #3.}}
\newcommand{\functionnn}[2]{\listrow{\fexprn{#1}$(t) = #2$}}
\newcommand{\adding}[1]{\listrow{Adding \exprn{#1} }}
\newcommand{\deleting}[1]{\listrow{Deleting \exprn{#1} }}
\newcommand{\error}{...Error!\\}
\newcommand{\errorr}[1]{...Error! \\ \> #1}


\newcommand{\commentcite}[2]{{%
    \renewcommand{\leftcite}{\relax}%
    \renewcommand{\rightcite}{\relax}%
    (#1, \cite{#2})}}
\newcommand{\egcite}[1]{\commentcite{\eg}{#1}}

%%% Local Variables: 
%%% mode: latex
%%% TeX-master: "total"
%%% End: 
